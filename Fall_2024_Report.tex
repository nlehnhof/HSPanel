\documentclass[journal]{new-aiaa} % Use the most recent AIAA journal class (automates all the formatting for you)

\usepackage[utf8]{inputenc} %use UTF-8 Encoding

\usepackage{graphicx} % Needed to insert images into the document
\graphicspath{{figures/}} % Sets the default location of pictures

\usepackage{amsmath} %math formatting package

\usepackage{longtable,tabularx} %table formatting packages

\usepackage{href-ul}

% Insert your Title Here
\title{FALL 2024 REPORT}

% Insert your Name and Information Here
\author{Nathan R. Lehnhof\footnote{Research Assistant, BYU FLOW Lab, Mechanical Engineering.}}
\affil{Brigham Young University, Provo, Utah, 84606}



\begin{document}

\maketitle % This generates the title and author information in the document.

%%%%%%%%%%%%%%%%%%%%%%%%%%%%%%%%%%%%%%%%%%%%%%%%%%%%%%%%%%%%%%%%%%%%%%
%                                                                    %
%             ABSTRACT: WRITE THIS AFTER EVERYTHING ELSE             %
%                                                                    %
%%%%%%%%%%%%%%%%%%%%%%%%%%%%%%%%%%%%%%%%%%%%%%%%%%%%%%%%%%%%%%%%%%%%%%
\begin{abstract}
	
	%%% --- Before/Motivation (WHY): --- %%%
	
	%% -- Context: Anyone Relates -- %%
	% Questions to answer:
	%    Why is the need so pressing or important? 
	%    What is the current state?



	%% -- Need: Readers Relate -- %%
	% Questions to answer:
	%    Why does anything need to be done? 
	%    Why is it important to the reader?



	%% -- Task: Author's Actions --%%
	% Questions to answer:
	%    What did you do to address the need?
	%    Why were you able to?



	%% -- Object: -- %%
	% Questions to answer:
	%    What does this paper present/cover? 
	%    Why?
	
	
	
	%%% --- After/Outcome (WHAT): --- %%%
		
	%% -- Findings: -- %%
	% Questions to answer:
	%    What did the work done yield/reveal?
	%    What did you find from the task?
	%    What are the results?
		
		
		
	%% -- Conclusion: -- %%
	% Questions to answer:
	%    What does it mean for the audience?
		
		
		
	%% -- Perspectives: -- %%
	% Questions to answer:
	%    What should be done next, future work, etc.?
	
	

\end{abstract}



%%%%%%%%%%%%%%%%%%%%%%%%%%%%%%%%%%%%%%%%%%%%%%%%%%%%%%%%%%%%%%%%%%%%%%
%                                                                    %
%                            INTRODUCTION                            %
%                                                                    %
%%%%%%%%%%%%%%%%%%%%%%%%%%%%%%%%%%%%%%%%%%%%%%%%%%%%%%%%%%%%%%%%%%%%%%

\section{Introduction}
\label{sec:intro}

%% -- Paragraph 1:  Motivation and Context for Transcribing Data -- %%
% Questions to answer: 
%    Why is the need so pressing or important? 
%    What is the current state?
Fall Semester of 2024 started off with two main focuses: helping Judd with his data validation (transcribing data tables) and a development project (Hess-Smith Panel Method). Transcribing the data is necessary to help Judd with his data validation. Through the data collected and reformatted, it is easier to use to confirm his findings through DuctAPE. Currently, we have finished all the transcribing.

%% -- Paragraph 2:  Context -- %%
The Hess-Smith Panel Method is more of a development project and is used in computational fluid dynamics for analyzing fluid flow about a body. By developing a Hess-Smith Panel Method, I have improved my workflow, coding ability, and understanding of computational fluid dynamics. It also gave me a greater understanding of the math and physics involved in analyzing airfoils, such as boundary conditions.

%% -- Paragraph 3:  Paper Outline -- %%
This report begins by focusing the process for transcribing the data tables, and then continues by following my journey through coding up a Hess-Smith Panel Method. With the Hess-Smith Panel project, I will first briefly go over the Hess-Smith Panel method theory, referencing the derivations in Dr. Ning's textbook. Then, I will walk through my pseudo code and first few drafts, highlighting issues and bugs that arose along with my process for overcoming them. Finally, I will present my final Hess-Smith Panel Method code, before repeating the above to explain my Hess-Smith Panel Method add-on (control surfaces).

%%%%%%%%%%%%%%%%%%%%%%%%%%%%%%%%%%%%%%%%%%%%%%%%%%%%%%%%%%%%%%%%%%%%%%
%                                                                    %
%                              METHODS                               %
%                                                                    %
%%%%%%%%%%%%%%%%%%%%%%%%%%%%%%%%%%%%%%%%%%%%%%%%%%%%%%%%%%%%%%%%%%%%%%

\section{Methods}
\label{sec:methods}

% A lot of your initial method details will probably come from content in your theory exploration assignment.  Feel free to copy and paste from that assignment as needed.

% Example of how to format your optimization problem in an equation format
\begin{equation} %using the equation environment will add a number and allow you to reference it.
\label{eqn:optimization_problem} %use this label to reference this equation
	\begin{aligned}
		\mathrm{minimize:~~} & (objective) \\ %Put your objective function here (minimize or maximize)
		\mathrm{with~respect~to:~~} & (design~variables) \\ % list out your design variables here
		\mathrm{such~that:~~} & (constraint1) \\ % constraints are usually < > =, ect.,
					   & (constraint2) \\ % so it helps to put them on multiple lines
					   %... etc.
	\end{aligned}
\end{equation}

\noindent where [explain the various symbols and what they mean here.]

%%%%%%%%%%%%%%%%%%%%%%%%%%%%%%%%%%%%%%%%%%%%%%%%%%%%%%%%%%%%%%%%%%%%%%
%                                                                    %
%                              RESULTS                               %
%                                                                    %
%%%%%%%%%%%%%%%%%%%%%%%%%%%%%%%%%%%%%%%%%%%%%%%%%%%%%%%%%%%%%%%%%%%%%%

\section{Results}
\label{sec:results}








%%%%%%%%%%%%%%%%%%%%%%%%%%%%%%%%%%%%%%%%%%%%%%%%%%%%%%%%%%%%%%%%%%%%%%
%                                                                    %
%                            CONCLUSION                              %
%                                                                    %
%%%%%%%%%%%%%%%%%%%%%%%%%%%%%%%%%%%%%%%%%%%%%%%%%%%%%%%%%%%%%%%%%%%%%%

\section{Conclusions and Future Work}
\label{sec:conclusions}



%% -- Paragraph 1:  Conclusion -- %%
% Questions to answer: 
%    What was the goal of this project?
%	 What was done to meet the goal?
% 	 What were the major takeaways?


%% -- Paragraph 2:  Future Work -- %%
% Questions to answer: 
%    What are the obvious next steps that were beyond the scope of this project? 
%    What else might be done in the future to improve this project or extend it?




%%%%%%%%%%%%%%%%%%%%%%%%%%%%%%%%%%%%%%%%%%%%%%%%%%%%%%%%%%%%%%%%%%%%%%
%                                                                    %
%                           BIBLIOGRAPHY                             %
%                                                                    %
%%%%%%%%%%%%%%%%%%%%%%%%%%%%%%%%%%%%%%%%%%%%%%%%%%%%%%%%%%%%%%%%%%%%%%

\bibliography{references}{} %all you need here is to put Bibtex entries in a file named references.bib, and things will automatically populate.
\bibliographystyle{aiaa} %the .bst file provides the information for this format automatically, just make sure it's in the same folder as this file.





%%%%%%%%%%%%%%%%%%%%%%%%%%%%%%%%%%%%%%%%%%%%%%%%%%%%%%%%%%%%%%%%%%%%%%
%                                                                    %
%                             APPEDICES                              %
%                                                                    %
%%%%%%%%%%%%%%%%%%%%%%%%%%%%%%%%%%%%%%%%%%%%%%%%%%%%%%%%%%%%%%%%%%%%%%

\appendix



\end{document}